% $Id: sec_mgmt.tex 3065 2012-10-23 12:31:42Z nkraft $

%This section describes the responsibilities and qualifications of the project personnel. PI Carver has wide experience in performing SLRs. First, in terms of publications, he has published two SLRs~\cite{GursimranSinghWalia_et_al_2009}~\cite{ByronJWilliams_et_al_2010}, has another SLR currently under review, and a multiple SLRs in preparation. Second, he taught an advanced graduate level course on SLRs, as described earlier in the proposal. Carver's primary roles in the project will be to provide overall direction, participate in the planning of the community interactions and guide the data analysis process. Co-PI Kraft, also has experience in performing SLRs, having published an SLR~\cite{JeremyRPate_et_al_2011}. He also has extensive experience in managing the development of software tools. His primary roles in the project will be to participate in the planning of the community workshops and provide input to the planning of tool development. Co-PI Hale comes from the Management Information Systems (MIS) domain and provides a different perspective on the project. We anticipate that his insight will be valuable in keeping the proper perspective. He will also participate in the planning of the community interaction and in the analysis of data from the workshops.

%In addition to the faculty participants, the project will include two PhD student participants. Both students have experience in SLRs based on their participation in PI Carver's SLR course. The first student will be a PhD student in MIS, whose dissertation work focuses on enabling various types of research processes in general, of which the SLR process is a specific example. The second student will be a PhD student in Computer Science who has experience in performing a SLR. The interaction between the MIS and CS PhD students will be quite useful to the project and valuable to those students.

%In this section we describe the qualifications and responsibilities of the project personnel.

PI Carver has extensive experience with the SLR process. He has planned and executed several SLRs, including two that are published~\cite{GursimranSinghWalia_et_al_2009,ByronJWilliams_et_al_2010}, one that is under review, and others that are in preparation. He has also taught an advanced graduate course on SLRs (Section~\ref{sub:SLR:course}).  PI Carver will oversee the project and coordinate the data analysis tasks. He will also participate in planning community interactions.

Co-PI Kraft also has experience with the SLR process. He has planned and executed two SLRs, including one that is published~\cite{JeremyRPate_et_al_2011} and one that is in preparation.  He also has extensive experience in leading the development of software tools in an academic environment and in managing the deployment of web-based systems in a production environment. Co-PI Kraft will coordinate workshop organization tasks. He will also participate in planning other community interactions and in planning the development of the software tools.

%Co-PI Hale brings a Management Information Systems (MIS) perspective to the project.
Co-PI Hale is from the Management Information Systems (MIS) domain and previously spent nine years with a joint appointment in a medical school.
He will provide input and feedback from an external perspective throughout the planning process.
He will also participate in planning community interactions and in data analysis tasks.

Other project personnel include two Ph.D. students, one from MIS and one from CS.
Both took PI Carver's SLR course and thus have have experience with the SLR process.
The MIS student's dissertation work is on enabling a variety of research processes, of which the SLR process is a specific example.
The CS student has planned and executed an SLR.
The interactions between the students from MIS and CS will benefit not only the students, but also the overall project.



% vim:syntax=tex
