% $Id: nsf_summary.tex 3089 2012-10-23 21:38:10Z nkraft $

{\bfseries\sffamily\slshape Keywords:} Systematic Literature Reviews, Software Engineering, Web-based Infrastructure

The Systematic Literature Review (SLR) process is a means for analyzing published evidence to draw conclusions about about a phenomenon of interest. The prevalence of empirical research in software engineering (SE) makes it a domain well-suited for the application of the SLR process. While the popularity of SLRs has been increasing within the SE community in recent years, there are still significant barriers to more widespread adoption. In particular:
\begin{itemize*}
   \vspace*{-3pt}
	\item{The process of systematically identifying relevant papers is a labor-intensive, largely manual process. Current SE literature databases lack adequate tools to assist in the extraction and analysis of a large body of literature. A researcher must manually manipulate search strings and parse through large numbers of search results, many of which are irrelevant}
   \vspace*{6pt}
	\item{There is little support for collaboration among multiple researchers. SLRs require the interaction of multiple researchers to review papers, extract data, and analyze the results. There is no infrastructure to support this critical task}
   \vspace*{6pt}
	\item{There is no mechanism to store the results of the SLR process to allow for reuse. Each paper analyzed during an SLR results in a completed data extraction sheet. There is currently no infrastructure to allow some or all of this data to be reused in subsequent SLRs}
   \vspace*{6pt}
	\item{There is no mechanism to allow SLRs to evolve over time. As new research is published, there needs to be a way for SLRs to evolve and stay current}
   \vspace*{-3pt}
\end{itemize*}

These barriers remain due to the lack of infrastructure support for the labor-intensive, largely manual process of performing an SLR. 
%The primary goal of this CI-P project is to \textbf{create an infrastructure to support the SLR process}.
The primary goal of this CI-P project is to \textbf{create an infrastructure to support the SLR process}.
%Our target community contains those SE researchers who already conduct SLRs as well as those SE researchers who will be more likely to conduct SLRs if the barriers are removed. The primary objectives of this CI-P proposal are to: 
The community targeted by this project comprises SE researchers who already conduct SLRs, as well as SE researchers who will be more likely to conduct SLRs if the barriers to adoption are removed.
%Building on our preliminary work,
The primary objectives of this CI-P project build on the investigators' preliminary work and are to:
\begin{itemize*}
   \vspace*{-3pt}
	\item{Evolve the community needs and proposed infrastructure identified in preliminary work}
   \vspace*{6pt}
	\item{Create a detailed plan for the development and deployment of the infrastructure}
	%\item{Elicit, validate, and prioritize the community needs for an SLR support infrastructure}
   %\item{Propose a new SLR support infrastructure based on those requirements}
	%\item{Create a detailed plan for the development and deployment of the infrastructure}
   \vspace*{-3pt}
\end{itemize*}

The {\bfseries\sffamily\slshape intellectual merit} of this project is in the fact that literature reviews, whether systematic or \emph{ad hoc} are an integral part of the research process. By enabling the execution of \emph{systematic} literature reviews, this infrastructure will help to ensure that researchers identify a complete and unbiased set of candidate papers when performing such a review.

The {\bfseries\sffamily\slshape broader impacts} of this project stem from lowering the barrier to performing SLRs. By removing the barriers currently faced by SE researchers, the proposed infrastructure will allow a larger portion of the SE community to participate in the conduct of SLRs. The infrastructure will be especially helpful to PhD students who will conduct a literature review as part of their thesis development. Because SLRs are often publishable as a stand-alone article, the infrastructure will also help increase the publication rate for these students.


% vim:syntax=tex
