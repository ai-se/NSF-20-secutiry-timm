% $Id: sec_prelim_tools.tex 3078 2012-10-23 19:15:47Z nkraft $

While we realize that the activities of this planning grant, specifically those defined in Section~\ref{sec:needs}, may provide additional inputs that could change the community needs identified in Section~\ref{sec:prelim:needs}, we do not anticipate any significant changes in the focus of the project. As a result, to illustrate the proposed infrastructure, this section provides an overview of our proposed solution. We plan to develop a web-based cyberinfrastructure that will enable: 1) coordination among multiple (possibly distributed) researchers, 2) community review of protocols, 3) automated interaction with different databases, 4) improved quality assessment, and 5) simple, repeatable data extraction. 

\subsection{Addressing Community Needs}
Our approach to addressing each of these needs is described in the remainder of this section.

\paragraph{Coordination among multiple authors} The web-based nature of the infrastructure will enable researchers who are not collocated to collaborate on execution of the same SLR protocol. The description of each of the requirements below also highlight features that enable multiple authors to collaborate on the same SLR using the proposed infrastructure. 

\paragraph{Community review of protocols}
The first and arguably most important step of the SLR process is the development, documentation, and validation of the protocol. The protocol includes: the motivation for the study, the research question(s), the sources for the primary studies, the search strings, the inclusion and exclusion criteria, the data extraction information, and the process details for each step of the execution phase. During construction of the protocol, the tool will also allow for the collaboration and feedback among the members of the SLR team. Once completed, the infrastructure will provide a mechanism for the public review of the proposed protocol and allow for community feedback. This feature will help provide researchers with confidence that the protocol is sound and will increase the chances of publication, if it is well-executed.

\paragraph{Automated interaction with databases}
To support the time consuming process of searching for and identifying primary studies the tool will provide a number of features. First, it will support both manual and automated searching of multiple databases. Second, it will allow for manual importation or automated retrieval of search results. Third it will help researchers create robust search strings by analyzing the provided keywords and suggesting alternatives, based upon the history of previous SLRs. Fourth, the system will adapt the standardized search strings defined in the protocol to the idiosyncrasies of known databases, where possible. Fifth, the tool will automatically detect duplicate papers in the various result sets, perhaps by using DOIs. Upon completion of the search phase, the tool will allow multiple researchers to evaluate the search results against the predefined inclusion and exclusion criteria. As each researcher's evaluation is captured independently, the tool will also provide inter-rater agreement ratings and isolate studies in need of further discussion among the SLR team. 

\paragraph{Improved quality assessment}
Once the final set of candidate studies is determined, the tool will support the quality assessment for each study based on previously defined criteria. As different types of research (e.g., lab experiments, field studies, etc.) should be evaluated based on criteria appropriate for the design, the system will allow researchers to classify and subsequently evaluate, using the appropriate criteria, each of the identified primary studies. Again, using the history of previous SLRs, the tool will help the researcher refine the quality assessment criteria by suggesting additional criterion that are related to those specified by the researcher. Once again, as each researcher's evaluation is captured independently the tool will provide information regarding agreement among the researchers and facilitate resolution of any conflicts.

\paragraph{Simple, repeatable data extraction}
For the final set of studies, the tool will provide a mechanism to facilitate data extraction. For strictly quantitative data extraction, the tool will perform an automated comparison to determine agreement among the researchers. To evaluate the extraction of qualitative information, the tool will provide a facility for third-party evaluation of the researchers' data extractions. As with other portions of the system, a means of evaluating agreement among the researchers and resolving conflicts will be made available. As the SLR moves into the analysis and then documentation stages, the tool will permit researchers to export citation information, in various formats, and all data collected during the extraction phase. 

\subsection{Enabling Future Research}
As indicated in the previous section, as each SLR performed utilizing the tool is completed, all of its data will be made available for inclusion in subsequent reviews, with proper attribution to the original researchers. We envision that this feature will support the establishment of research communities focused on a given topic or domain. As new research is completed, studies can be added to the existing knowledge base and processed for inclusion in updated SLRs.

While the infrastructure initially targets the research community, inclusion of practitioners in the target community could provide additional possibilities. Practitioners may provide expert opinions and evaluations of SLR topics, or provide guidance related to important research questions that need to be answered. Furthermore, the system may facilitate the transfer of knowledge gained from research to practitioners in the field.

% vim:syntax=tex
