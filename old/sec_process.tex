To help ensure the rigor required to conduct an unbiased, repeatable, and thoroughly documented SLR, the researchers perform three phases.
At each phase of the process, multiple researchers work independently and then meet to resolve any differences through discussion~\cite{Kitchenham:04, Kitchenham_Charters:07, Biolchini-etal:05}.
First, to provide transparency, in the \textit{planning} phase, the researchers develop a formal protocol that explicitly defines a plan for:
\vspace{-8pt}
\begin{enumerate}
	\item A comprehensive literature \textbf{search} to identify candidate primary studies of interest, including an explicit description of the research questions, the databases to be searched, and the search string to be used.
	\vspace{-4pt}
	\item A culling of the candidate studies to \textbf{select} those that are relevant, including detailed inclusion and exclusion criteria to make the selection more objective.
	\vspace{-4pt}
	\item A \textbf{quality assessment} of each selected study using a predefined quality assessment rubric.
	\vspace{-4pt}
	\item \textbf{Data extraction} from each selected study using a data extraction form to ensure consistency across all included studies.
	\vspace{-4pt}
	\item A \textbf{synthesis} of the extracted data to answer the research question(s).
\end{enumerate}
\vspace{-4pt}
Second, in the \textit{execution} phase, researchers perform each of the steps in the defined protocol.
Finally, in the \textit{documentation} phase, researchers explicitly document the exact process followed and the findings from the synthesis.  

While research reports typically describe the SLR process as being sequential, our interactions with SLR researchers reveal it to be much more iterative~\cite{Carver-etal:13, Hassler-etal:14}.
As shown in Figure~\ref{figure-SLR-Process}, the results of one step may require re-planning of later steps or even re-planning and re-execution of previous steps.
Feedback from SLR researchers suggests that an ongoing monitoring process helps to ensure that the research team is in agreement and that the SLR is proceeding according to a valid, documented plan.
Each step is preceded by a form of stage-gate that serves as a quality checkpoint.
Therefore, prior to each step, the SLR team reviews and revises the plan as needed.
This monitoring allows the SLR team to maintain confidence in the results. 

Because SLR teams are often distributed, there is a need for appropriate communication facilities.  
The SLR team often must interchange large data sets -- including full articles at early phases.
In addition to the resources required to perform the individual SLR steps, resources must be expended on management and coordination of the team.
Thus, the production of a single SLR is a difficult and resource intensive effort.

As a result of the formal planning and systematic process described above, the SLR approach is fundamentally different from the \textit{ad hoc} approach in that SLRs:
\vspace{-8pt}
\begin{itemize} 
	\item Reduce the potential of accidentally omitting important papers;
	\vspace{-8pt}
	\item Are more likely to be executed by a team of researchers working collectively;
	\vspace{-8pt}
	\item Are more likely to provide independent, stand-alone value as a research result; and
	\vspace{-8pt}
	\item Are more likely to be replicated and updated by other researchers.
\end{itemize}
\vspace{-4pt}