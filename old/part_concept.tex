A \textit{Systematic Literature Review (SLR)} is a rigorous, standardized process by which researchers identify and analyze published evidence to draw broad conclusions about a phenomenon or research question.
The systematic, comprehensive, and reproducible nature of SLRs leads to findings that are less prone to bias and omissions than are s typical of the traditional \textit{ad hoc} literature review.
Given the frequency of empirical studies in software engineering (SE), the domain is well-suited for SLRs.
Although SLRs are increasingly recognized as vital to the SE discipline, the lack of infrastructure to support this effort-intensive, largely manual process results in barriers that inhibit their widespread adoption.
Building upon the results of our CI-P grant (NSF-1305395), the primary goal of this CI-NEW project is to build infrastructure to reduce these barriers and increase adoption of SLRs in SE.
For the sake of this proposal, we define \textit{infrastructure} to include both the back-end data storage and interchange, as well as the front-end user tools that support specific SLR steps.
Our target community comprises SE researchers who already conduct SLRs, SE researchers who will be more likely to conduct SLRs if the barriers are removed, SLR tool builders, and industry consumers of SE research results. 
The primary objectives of this CI-NEW project are to: 
\begin{itemize*}
	\vspace{-6pt}
	\item Build an extensive SLR support infrastructure;
	\item Ensure the infrastructure is flexible enough to incorporate existing and new tools; and
	\item Validate the usefulness of the infrastructure with members of the target community.
\end{itemize*}
\vspace*{-4pt }
