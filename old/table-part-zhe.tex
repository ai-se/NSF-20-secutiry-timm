\begin{table}
\caption{A sample of the manual and automatic {\em methods} available for exploring the SE literature.
The point of this proposal is to offer the infrastructure 
that lets  SE
community decide   how to make best use of these methods.
Note that this list is incomplete since researchers are
regularly proposing new approaches. Note also that while we present this list as a {\em linear} process (running from top to bottom), 
SLRs are usually a {\em cyclic process} where analysts make multiple iterations  across  these methods.}\label{tbl:overview}
\begin{center}
 {\small
  \begin{tabular}{p{1.5cm}|p{5.5cm}|p{8.3cm}}
    \rowcolor{gray}   & \textcolor{white}{Primarily manual methods} & \textcolor{white}{Automated AI-based text mining methods }  \\
    \hline
    1a. 
    \newline Define \newline research \newline question &   Read many papers (a very slow process)~\cite{kitchenham2004evidence,keele2007guidelines,marshall2013tools}. If there was a repository on prior questions we could explore them. &
    \bii
    \item {\em Dcluster:} Clustering documents with (e.g.) topic modeling~\cite{Mathew_2018};
    \item {\em Qcluster:} Cluster old queries to find common patterns in queries (and results);
    \item Find the gaps (clusters with low frequencies).
    \eii  \\\hline\rowcolor{blue!10}
    1b. \newline Identify  keywords& Extract keywords from research question topics and known related literature &
    \bii
    \item Extract candidate words from {\em Qclusters};
    \item Use feature selection (e.g. TF-IDF or RELIEF on support vectors);
    \item Auditing the keywords with synonym discovery (PCA~\cite{Wold1987Principal}, Word2Vec~\cite{mikolov2013efficient}). 
    \item Use of information extraction tools~\cite{Cruzes2007Automated}. 
    \eii \\
    \hline
    1c.\newline Create search string  & 
    \bii
    \item
    Use synonyms of keywords to identify terms for search string.
    \item
    Using subject matter experts (hard, they are hard to find).
    \item
    Use of gold standard to refine the search string~\cite{zhang2011empirical}.
    \eii &
    \bii
    \item Construct most informative search strings using  active learning~\cite{Yu2018,Yu2019};
    \item Using  SVMs, cull terms  explore those found most often
    far from hyperspace boundary;
    \item Apply Word2Vec~\cite{mikolov2013efficient} to extend query with keyword synonyms (as done in \cite{georgeLN}).
    \eii \\
    \hline\rowcolor{blue!10}
    2. \newline Search databases \cite{kitchenham2004evidence,keele2007guidelines}& \bii
    \item
    Snowballing~\cite{jalali2012systematic}.
    \item
    Use various permutations of search string in each database (sometimes requiring multiple searches per database) 
    \eii & AI
    not required for this task. However, with automatic tool supports, the time required for this task can be  greatly reduced. 
    \\
    \hline
    3. \newline Select \newline relevant papers&Current practice: a mostly manual process~\cite{kitchenham2004evidence,keele2007guidelines,hassler2016identification}&
    \bii
    \item Use active learning trained/updated on human review results to recommend what next to be reviewed~\cite{Yu2018,Yu2019}.
    \item Try initializing queries with results from {\em Qcluster}~\cite{Pan2010A}. 
    \item Explore using information other than title and abstracts, e.g. use full-text information or summary of text~\cite{summary1,summary2,summary3} 
    \eii\\
    \hline\rowcolor{blue!10}
    4. \newline Assess\newline quality  & Some checklists (number of subjects, type of studies, e.g bad smells)~\cite{kitchenham2004evidence,keele2007guidelines} & Not currently explored
    by AI tools~\cite{marshall2013tools}. This might be a matter of entity recognition~\cite{manning-EtAl:2014:P14-5} over checklist but really cannot be definitive at this time. %This is an area where we hope some share data and shared conventions on how to integrate tools would enable more experiment and more tools in the future. 
    \\    \hline
    5. \newline Extract data & Manual process: reading the the actual papers. E.g. run over a pdf and add mark ups over the text (so you can localize where you look)~\cite{kitchenham2004evidence,keele2007guidelines}.  & Topic modeling and visualization tools might help here~\cite{Felizardo2010An,Torres2012Automatic}\\
    \hline\rowcolor{blue!10}
    6. \newline Synthesis of results &  Manual work with help using  tools~\cite{kitchenham2004evidence,keele2007guidelines,marshall2013tools}. & Visualization might be able to help this process~\cite{Cruzes2007Using,Felizardo2011Analysing,Felizardo2010An} \\
    \hline
    7. \newline Summary of papers&   & Not done right now. But summarizing of technical artifacts is an emerging natural language technology~\cite{gambhir2017recent} so we can look to much improvement in this area, in the near future. \\\hline
  \end{tabular}
 
}
\end{center}


\end{table} 