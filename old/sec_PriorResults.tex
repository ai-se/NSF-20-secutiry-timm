This proposal builds on the prior work of PI Carver from his planning grant. 
\underline{(a)} CNS-1305395, 10/1/13-9/30/15, \$100,000, 
\underline{(b)} CI-P: Advanced Systematic Literature Review Infrastructure for Software Engineering,
\underline{(c)} The \textbf{Intellectual Merit} of this work provided the information necessary to develop a plan for building an infrastructure to support SLRs, which are an integral part of the research process.
\underline{(c)} The \textbf{Broader Impacts} of the work included the development of such an infrastructure will lower the barriers and increase the quality and quantity of SLRs produced by the SE community resulting in more widespread dissemination of research results.
\underline{(d)} This grant supported the work of two PhD students, one of which has completed his degree, and has produced two conference papers~\cite{Carver-etal:13, Hassler-etal:14} and three journal papers~\cite{Hassler-etal:16,AlZubidy-etal:2017,Al-Zubidy2019}.
\underline{(e)} Data from this project is housed in the project website~\cite{workshop}.
\underline{(f)} N/A.


This proposal is the next logical step in the Co-PIs's work on analytics and text mining
{PI Menzies} has worked extensively in that arena.
\textbf{Dr. Tim Menzies} has several prior NSF  grants relevant to this work.  
For example, one  prior grant was:
 \underline{(a)}~CCF-1017330 (2011 to 2015,  \$350,000),
\underline{(b)}~``SHF: Small: Collaborative Research: Better Comprehension of Software Engineering Data''. 
\underline{(c)}~The {\bf intellectual merit} of this  grant
was  to define the data, evaluation methods, challenge problems and baseline results for this style of data mining.  The {\bf broader impact} of this  grant was to  inspired literately hundreds of subsequent papers,
 by other researchers.
These  grants generated \underline{(d)}~24 publications~\cite{%
Bavota2012a,%
Bavota2013,%
Bavota2012b,%
Haiduc2010a,%
Haiduc2013a,%
Haiduc2012a,%
Haiduc2013b,%
Ohlemacher2011b,%
Scanniello2013,%
Scanniello2011,%
me11m,% 
me13a,peters12a} and 
partially supported five masters students and three Ph.D. students. \underline{(e)}~Data from that work is now housed at the SEACRAFT publicly accessible repository~\cite{seacraft} and lessons learned from that work was widely disseminated in the book ``Sharing Data and Models in SE''~\cite{Menzies:2014:SDM:2930830}. \underline{(f)} N/A.  



More recently, Co-PI Menzies worked on \underline{(a)}~CCF-1302216, 2013-2107, \$271,553;   \underline{(b)}~``SHF: Medium: Collaborative: Transfer Learning in Software Engineering''. 
 \underline{(c)}~The {\bf intellectual merit} of that work was to
define novel methods for sharing data, many of which were the precursor to the methods of this proposal.  That work generated the publications  \underline{(d)}~\cite{krishna2018bellwethers,PetersML15,krishna16,he13,Me17,fu2016tuning,krishna2017learning}, that has been privatized yet still  capable of building effective models
for prediction and planning purposes.
The {\bf broader impact} of that work was to
enable a new kind of open science-- one where all data is routinely shared and is capable of building effective models no matter if it is obfuscated for security proposes.
The methods of this project, while targeted at software engineering, could also be applied to any other data intensive field.   
 \underline{(e)}~Data from that work is now housed at the SEACRAFT publicly accessible repository~\cite{seacraft}. That work  funded two Ph.D.s at NCSU. \underline{(f)}
N/A.  

%There are other grants that could be listed here but as we understand recent changes to NSF grant guidelines, the above are all that is allowed/required.