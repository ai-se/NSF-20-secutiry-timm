
\textbf{The proposed SLR infrastructure tool set provides the means by which teams of researchers produce unbiased, repeatable comprehensive literature reviews and are able to evaluate and synthesize results of multiple primary studies.} 
Thus the quantity and quality of SE research is amplified by providing timely and less resource intensive means to:
\vspace{-8pt}
\begin{itemize}
	\item Ground pre-existing theoretical foundations to the research questions and experimental operationalizations.
	\vspace{-4pt}
	\item Evaluate and synthesize results from reviewed primary studies to provide guidance regarding best practices, and empirically grounded evidence to the discipline benefiting both researchers and practitioners.	
	\vspace{-4pt}
	\item Iteratively and periodically renew the results of previously executed SLRs.
\end{itemize}
\vspace{-4pt}
	
The following are key benefits that will be derived from the SLR infrastructure for specific components of the community.

\paragraph{Researchers and SLR Authors:}
A compelling anticipated result of using the SLR infrastructure tool sets is that researchers and SLR authors will be able to better ground new research initiatives in existing literature to ensure that replicated studies are well conceived.  
Moreover, when variants of existing research approaches are conducted, the research results will be able to be integrated with prior studies to make more powerful/pervasive combined inferences.  
Researchers will be able to use the SLR Infrastructure collaboratively to plan, conduct, and document SLRs. 
Enabled by the infrastructure tools, researchers will be able to coordinate their efforts, individually or in a team, at each step in the process.   
Infrastructure tools will enable research teams to be geographically distributed, and work over time.  
That is, the infrastructure will provide persistence of searches, protocols, and search results. 
This persistence will allow teams to allocate work across the SLR process steps, accomplish individual tasks and then synthesize results. 
Moreover researchers will be able to leverage previous SLR protocols and exemplars (with permission).  

Through the monitoring functionality built into the infrastructure, at each step in the SLR process discoveries not envisioned during the initial planning step will be facilitated through explicit coordination and revised planning.   
This functionality will support the iterative nature of the SLR process mentioned previously.   
As a special case of researchers and authors, the infrastructure will provide guidance to \textit{PhD students and other novice researchers}. 
Individually and in teams, these novices will be able to assemble protocols and templates to plan, conduct, and document their literature reviews. 
Thus, they will draw upon previous research to identify sources of literature and possible concepts to explore. 
The SLR infrastructure monitoring and coordination tools will allow supervisors/chairs to provide feedback as needed, and optionally approve each step of the SLR process. 
Through this guidance their research can be more complete and unbiased, thus leading to more impactful research results and conclusions. 


\paragraph{SLR Tool Developers:}
The open nature of the proposed SLR infrastructure allows existing and future SLR tool developers to research, develop and deliver state-of-the-art tools. 
SLR tool developers can focus on the specific SLR process steps that will provide the most return-on-investment while developing these tools.  
That is, the SLR tool developers' research goal is to provide the highest quality results that consume the least author effort.  
Moreover, research can be conducted on effective interface design, and efficacy of the SLR process including the activities such as planning, monitoring, coordinating, and executing of tasks.  

%\paragraph{Software Engineers:}
%\note{Does this really belong here? Are we enabling research here?}
%Beyond direct use of the SLR infrastructure tools by researchers and novices, this infrastructure will enable software engineers to directly influence their practice. 
%A fundamental goal of SLRs is based on an identified or perceived weakness in practice to extract, evaluate and synthesis empirically-based leading processes. Thus the proposed infrastructure will provide:
%\vspace{-8pt}
%\begin{itemize}
%	\item More applied research that is directly commissioned by (and applicable to) practitioners, and
%	\vspace{-4pt}
%	\item More basic research as substantive dialogs generate fundamental questions to be researched. 
%\end{itemize}