\begin{center}
{\bf \TITLE\\
Jeffrey C. Carver (U. Alabama) and Tim Menzies (NC State)}
 \end{center}
 
The SE research literature has become incomprehensibly large. In the last 15 years, over 16,000 authors have written over 15,000 research papers just at major SE conferences. Yet, these papers received in average less than one citation per paper. This is troubling since it means most research is not cited, not used by practitioners, forgotten, and hence is wasted effort. For organizations such as NSF, this also means that much funded research never receives the public recognition and credit it deserves.

Given the latest advances in AI text mining, anyone exploring the SE research  literature might be able to simplify literature reviews using numerous automatic methods. But there are now so many such automatic methods that, without appropriate infrastructure, it is hard for SE researchers and practitioners to understand them (let alone try to apply them). Accordingly, we propose augmenting standard SLRs (systematic literature reviews) with SAInT (the SLR Artificial Intelligence Toolkit), an AI-enabled infrastructure for reading the literature. We propose SAInT as a community infrastructure project to support researchers in conducting better literature reviews. Our goals are to allow a large community of researchers and industrial practitioners to conduct 100+ SLRs   with these tools. We know that best results from combining artificial and human intelligence come from  each doing what each does best. For example, automatic AI methods are useful for looking over large volumes of data (something which humans find tedious and/or overwhelming). Human intelligence, on the other hand, is best at offering deep insights on particular examples. Hence SAInT uses  AI  (to prune out most of the examples) and human intelligence (to offer insight on the pruned results).


 
 \noindent
\underline{{\bf INTELLECTUAL MERIT:}}
This work will allow better synthesis  of more research from a wider group of researchers, thus enabling extensive and rapid innovation in many areas of SE.
Also, this research relates to the usability of AI methods. AI tools are all very well, but how should a large community make best use of these newly emerged and experimental techniques? SAInT will answer this question, while at the same time allowing more researchers and industrial practitioners to find more relevant research, faster. 

SAInT also addresses another issue. There is no consensus on what is a "valid" use of AI text mining tools for SLRs (where "valid" means that researcher1 will accept    conclusions made by researcher2 using these AI text mining). Such a consensus is essential to provide scientific repeatability, improve coverage, reduce human labor and errors, and allow for the iteration and improvement of SE knowledge. The community using SAInT will evolve a consensus of what is a "valid" AI-enhanced SLR analysis.
  

\noindent
\underline{{\bf BROADER IMPACTS:}}
We focus on an issue of tremendous economic importance- the creation of better quality software using state-of-the-art research results. As a result, this work will increase America's ability for industrial and academic innovators to conduct more scientific studies via computational means.

The PI and co-PI teach senior-level and graduate-level empirical SE, and data  mining classes (and in those data mining classes, all the case studies come from software  engineering). All of the technology developed in this proposal will become case study material for those subjects.  

PI Carver will build on his track record of supporting PhD students from historically under-represtented minorities in Computer Science. Given University of Alabama's geographical location, he will also invest efforts to recruit high-quality PhD students from regional HBCUs and other minority-serving institutions.

Co-PI Menzies will continue his established tradition of graduating research students for historically under-represented groups. This work will inform the curriculum of the his annual NSF-funded REU (research experience for undergraduates) work on the "Science of Software" (in this program, places are reserved for students from traditionally under-represented areas; e.g. economically challenged regions of the state of North Carolina) and/or students from universities lacking advanced research facilities). While some of the concepts of this grant would be too advanced for that group, some of the simpler text mining concepts and case studies would be suitable for lectures and REU projects.

\noindent \underline{{\bf KEYWORDS:}} software analytics, transfer learning, prediction