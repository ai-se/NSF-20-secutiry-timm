We intend to perform both formative and summative evaluation of the proposed infrastructure. 
This section describes each of those evaluations in detail.

\subsection{Formative Evaluation}
Formative evaluation will help ensure that we are developing the most appropriate infrastructure for the various stakeholders described in Section~\ref{sec:community:needs}.
During development of the infrastructure, we will interact with key stakeholders (especially those who have provided support letters) to ensure that the infrastructure will meet their needs.

First, we will have ongoing interactions with members of the SLR tool building community.
These developers are key to the formative evaluation because we must ensure that the design of our infrastructure can enable the integration of external tools.
We will seek input from the tool builders both early in the process, to inform our design, as well as throughout development to ensure the infrastructure is meeting their needs (see attached letters).

Second, we will release early versions of the infrastructure to our collaborators who have expressed interest.
We have a number of SLR authors who have agreed to provide feedback on early versions of the infrastructure to help ensure that the end-user functionality is appropriate (see attached letters).
Specifically, we will ask these users to comment on whether the tools meet their needs in each phase in the SLR process.
We will gear our requests towards gathering feedback that can inform the future phases of development. 

Finally, mid-way through the project, we will conduct a community workshop. 
The goal of this workshop will be to ensure that we are on course to realize value from the implementation of the SLR infrastructure. 
During the workshop we will demo the current infrastructure and give workshop attendees an opportunity to interact directly with it. 
The workshop will provide opportunities for authors and tool developers to discuss their perspectives and provide feedback once they have reviewed the status of the SLR infrastructure tools.  
Our desire is to ensure the implementation is providing functionality to meet their needs.
We will specifically ask the workshop attendees to provide feedback to guide the remainder of the development effort.

\subsection{Summative Evaluation}
Throughout development, the summative evaluation will quantify the adoption and perceived usefulness of the new infrastructure.
We will gather data in four ways:
(1)~we will periodically send short surveys to infrastructure users to measure their acceptance and perceived usefulness of the infrastructure;
(2)~we will gather usage statistics, including the number of users, the number of times various features and tools are used, how much data is stored in the repository, and how many SLRs are completed using the tool;
(3)~we will gather insights from attendees of our second community workshop, to be held at the end of the grant period, to help ensure that SLR tool developers and authors find value in SAInT, which will be the key success trigger to ensuring its sustainability beyond the grant period; and
(4)~we will provide a citable URL so we can track how many SLRs are completed with the aid of the infrastructure. 

The second community workshop will be similar to the first community workshop.
We will invite active members of the SLR community (both authors and tool builders) to participate in this interactive workshop.
We will use this time to train attendees on how to use SAInT and how to build additional tools to plug into SAInT.
In addition, the feedback data gathered during this workshop will provide input to the summative evaluation.
