The next section of this proposal
offers some design principles for {\IT}.
These principles were learned
as part of planning grant NSF CNS-1305395 where PI Carver conducted a community survey and a series of community workshops to identify the SE community needs for SLR infrastructure~\cite{hassler2014outcomes, carver2013identifying, hassler2016identification}.
The rest of this section describes the results of that planning grant.
\vspace{8pt}

\subsection{Results of Community Survey}
In preparation for the Community Workshops described in the following sections, we surveyed published SLR authors to understand their SLR process, the difficulties they encountered, and the areas most in need of tool support.
Based on the 59 responses we identified four primary areas in need of infrastructure support~\cite{Carver-etal:13}:
\vspace{-8pt}
\begin{itemize}
	\item Systematically identifying relevant papers from the literature;
	\vspace{-4pt}
	\item Conducting collaborative SLRs;
	\vspace{-4pt}
	\item Storage of extracted data from papers; and
	\vspace{-4pt}
	\item Evolution of SLRs over time.
\end{itemize}

\subsection{Overview of Community Workshops}\label{sect:work}
\noindent
To gather additional community input, we ran the following workshops:
\bi
\item \textbf{ESEIW'13}: 
The first workshop, held concurrently with the 2013 Empirical Software Engineering International Week (ESEIW), focused on identifying and ranking barriers to the execution of the SLR process.  
Utilizing a nominal group technique~\cite{Delbecq-etal:75,Lethbridge-etal:00}, the ESEIW workshop participants generated over 200 individual ideas, which they consolidated into 45 composite barriers in the SLR process. 
Independently, the EASE workshop participants identified 28 composite barriers (21 matched barriers from the ESEIW workshop and 7 new ones focused on tool features).
\item \textbf{EASE'14}:
The second workshop, held in conjunction with the 18th International Conference on Evaluation and Assessment in Software Engineering (EASE), validated the results of the first workshop and served primarily to identify and rank the requirements for an SLR infrastructure.
The workshop participants were all researchers actively engaged in the production of SLRs in the SE domain.  
The 30 attendees represented a globally diverse slice of the community with representatives from North America, South America, and Europe.  
\ei
% \subsection{Analysis of Existing Tools}
At those workshops, participants reviewed a range of
SLR support tools.
They found that SLRs still suffer from a lack of complete tool support.
According to a mapping study~\cite{Marshall-Brereton:13}, 11 SLR-related tools are described in the SE literature. 
Most of these tools (7 of 11) target the selection of studies to include in the review~\cite{MS:11,MS:13,MS:14,MS:21,MS:22}, the extraction of data~\cite{MS:12,MS:18,MS:21}, or the synthesis of data~\cite{MS:12,MS:16,MS:17,MS:20}.
Three of the other tools~\cite{MS:15,MS:19,MS:23} target the entire SLR process.
The remaining tool~\cite{MS:24} targets the identification of candidate studies.
Unfortunately, most of the tools identified by the mapping study~\cite{Marshall-Brereton:13} are either in early stages of development or are not freely available.
Furthermore, only two of those tools~\cite{MS:18,MS:23} have been evaluated independently~\cite{Marshall-Brereton:13}.

Starting from the list in the mapping study~\cite{Marshall-Brereton:13}, we selected the three tools that were specifically focused on SLRs, targeted multiple phases of the SLR process, and were freely available (SLuRP, StART, and SLR-Tool).
We also identified two additional tools not covered in the mapping study (SLRTOOL and Parsifal).
(Note that we excluded ReVis and PEx from our evaluation because there are designed to support single phases of the SLR process. However, we will include those tools in our construction of SAInT, where appropriate.)
Using information from the literature and our own evaluation of the tools, we documented each tool's features and limitations~\cite{Al-Zubidy-Carver:14}.
Based on the detailed results of the Community Workshops, we analyzed each tool's features to determine whether they covered any or all of the desired features.
We iterated the results of our analysis with each tool's authors to ensure we had accurately characterized them.
Table~\ref{table-tools} summarizes our findings and shows that none of the tools fully support the requirements for the entire SLR process.

\begin{table}
	\centering
	\caption{Tools - \ding{51} = fully covered; \ding{119} = partially covered; \ding{109} = not covered. The important observation to be made from this table is that all the current SLR tools
	offer partial support for parts of the SLR process. This table lends suppport
	to the design premises of {\IT}; i.e. (a)~that
	no single tool offers support for the entirety of SLRs; and (b)~best results come from combining
	the capabilities of multiple tools.}
	\resizebox{\textwidth}{!}{
	\begin{tabular}{|m{1.8cm}|m{1.4cm}|m{1.5cm}|m{1.5cm}|m{1.4cm}|m{1.3cm}|m{1.5cm}|m{2cm}|m{1.5cm}|m{2.1cm}|}
%	\begin{tabular}{|p{1cm}|c|c|c|c|c|c|c|c|c|}		
\hline
~                               & Planning & Searching & Selection & Quality Assessment & Data Extraction & Data Synthesis & Presentation & Report Generation & Coordination \\ \hline
\rowcolor{blue!10} Connection to \tbl{overview} &Rows 1a,1b,1c&Row 2&Row 3&Row 4&Row 5&Row 6&\multicolumn{2}{c|}{Row 7}\\\hline
SLuRP                       & \centering\ding{119}        & \centering\ding{119}         & \centering\ding{119}         & \centering\ding{119}                  & \centering\ding{119}               & \centering\ding{119}              & \centering\ding{119}            & \centering\ding{119}                 & \ding{119}           \\ \hline
StArt                          & \centering\ding{119}        & \centering\ding{119}         & \centering\ding{119}         & \centering\ding{119}                  & \centering\ding{119}               & \centering\ding{119}              & \centering\ding{119}            & \centering\ding{119}                 & \ding{109}            \\ \hline
SLR-Tool                    & \centering\ding{119}        & \centering\ding{119}         & \centering\ding{119}         & \centering\ding{119}                  & \centering\ding{119}               & \centering\ding{119}              & \centering\ding{119}            & \centering\ding{119}                 & \ding{109}            \\ \hline
SLRTOOL 				& \centering\ding{119}         &
\centering\ding{109}         & \centering\ding{119}         & \centering\ding{119}                  & \centering\ding{119}               & \centering\ding{119}              & \centering\ding{119}            & \centering\ding{119}                 & \ding{119}            \\ \hline
Parsifal & 						\centering\ding{119}        & \centering\ding{119}         & \centering\ding{119}         & \centering\ding{119}                  & \centering\ding{119}               & \centering\ding{119}              & \centering\ding{119}            & \centering\ding{119}                 & \ding{119}           \\ \hline
\end{tabular}
}

%51 = F
%109 = N
%119 = P

	\label{table-tools}
\end{table}

The problem of incomplete SLR tool support is not restricted to just the SE community.
The medical community often uses the following tools to support SLRs.
Note that neither tool provides much support for the planning or execution phases of the SLR process:
\bi
\item
RevMan~\cite{RevMan} focuses primarily on the documentation phase of SLRs performed under the guidelines of the Cochrane Collaboration. 
It facilitates the preparation of formatted tables and version tracking. 
\item
Archie~\cite{Archie} is the central repository for RevMan and is used to store completed reviews. 
\ei
Review of the medical literature and interaction with an MD who participate in our first workshop showed that the medical community faces many of the same problems faced by the SE community.
Many of these barriers relate to the difficulties faced during searching~\cite{Bouamrane-etal:11, Frunza-etal:10, Young-Ward:01, Zwolsman-etal:13, Zwolsman-etal:12, Lai-etal:10}.
\vspace{8pt}

\subsection{Requirements for an SLR Tool}\label{tion:results}
Following on from the above community engagement exercises,
and using the results of the surveys and discussions at those  workshops, we were able to identify a number of detailed requirements for an SLR infrastructure.
We abstracted those detailed requirements into the following list of high-level requirements to drive the design and development of {\IT}.
\be
		\item Support task automation -- substitute algorithmic and computer cycles for human work
		\item Support automated guidance (phase protocol execution support)
		\item Support iteration within and among SLR Phases
		\item Support project collaboration and coordination of work teams
		\item Support monitoring of progress and quality
		\item Support revised planning
		\item Support data interchange across all SLR phases
		\item Support existing and future tools through workflow integration and data interchange
		\item Support an open architecture
\ee
The next section argues that these requirements can be addressed via a novel infrastructure combing some systems utilities with the manual and automatic methods of \tbl{overview}.
\vspace{8pt}