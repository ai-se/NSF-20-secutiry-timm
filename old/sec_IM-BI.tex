\newcommand{\SHAW}{{\em {\sffamily BRIDGE}}}
\subsection{Intellectual Merit}
The proposal aims to package and offer to the broader SE community
innovated next-generation AI text mining tools. 

This work will allow better synthesis  of more research from a wider group of researchers, thus enabling extensive and rapid innovation in many areas of SE.
Also, this research relates to the {\em usability} of AI methods.
AI tools are all very well, but how should a large community make best use of these newly emerged and experimental techniques? {\IT} will answer this question, while at the same time allowing
more researchers and industrial practitioners to find more
relevant research, faster. 

{\IT}  also addresses another   issue.
There is no consensus on what is a ``valid'' use of AI text mining tools  for SLRs (where ``valid''  means  that  researcher1  will  accept     conclusions  made  by  researcher2  using  these AI text mining).   Such  a  consensus  is 
  essential  to  provide  scientific  repeatability,  improve  coverage, reduce human labor and errors, and allow for the iteration and improvement of SE knowledge.
  The community using {\IT} will   evolve a consensus of what is a ``valid'' of AI-enhanced SLR analysis.
  

Also, another meritorious  aspect of this work was set out in the {\SHAW} paper that warned that 
when practitioners cannot access conclusions from research, then
that is a   symptoms of a researcher community that is poorly defining its research outputs. By making more literature accessible to more people,
this infrastructure will enable a much greater impact of our research on
industrial practice.
\vspace{8pt}

 

\subsection{Broader Impact:} 
We focus on an issue of tremendous economic importance- the creation of better quality software using state-of-the-art research results.  
As a result, this work will increase America's ability for industrial and academic innovators to conduct more scientific studies via computational means.

The  PI and co-PI teach senior-level  and  graduate-level  empirical  SE,  and  data  mining  classes  (and  in  those  data mining  classes,  all  the  case  studies  come  from  software  engineering).   
All  of  the  technology  developed in this proposal will become case study material for those subjects.  

This project will have beneficial diversity impacts:
\bi
\item
PI Carver will build on his track record of working with historically under-represented students in Computer Science.
He has graduated two female PhD students and is currently working with two more (one of whom will graduate this year).
He also graduated the first African-American PhD students from the Computer Science department at Mississippi State University (where he was prior to moving to the University of Alabama).
Given it's geographical location, the UA College of Engineering has a Multicultural Engineering Program which focuses on recruitment/retention and supports UAs participation in the Southeastern Consortium for Minorities in Engineering.
PI Carver will seek out qualified students from regional HBCUs and other minority-serving institutions.
\item
Co-PI Menzies will continue his established tradition of graduating research students for historically under-represented groups.   
This work will inform the curriculum of the PI's annual NSF-funded REU (research experience for undergraduates) work on the ``Science of Software'' (in this program, places are reserved for students from traditionally under-represented areas;  e.g.  economically challenged regions of the state of North Carolina) and/or students from universities lacking advanced research facilities). 
While some of the concepts of this grant would be too advanced for that group, some of the simpler text mining concepts and case studies would be suitable for lectures and REU projects.
\ei
\vspace{8pt}


% \paragraph{7.3. Integration of Teaching and Research}
% Much of the research in this project will also be integrated into a
% classroom environment.  The PI teaches senior-level and graduate-level
% SE, and software analytics   classes (and in those data mining
% classes, all the case studies come from software engineering). All of
% the technology developed in this proposal will become study
% material for those subjects.  Replication studies are especially ripe
% for classroom projects.  Also, through our publications and conference
% work, we would publicize these tools as widely as possible with the
% intent of supporting a broader community working with this approach.

% \paragraph{7.4 Participation of Underrepresented
%   Groups} The PI will continue their established tradition of
% graduating research students for historically under-represented
% groups.  PI Menzies' last two Ph.D. students were an African-American
% women and a gentleman from an economically disadvantaged region
% in central Pennsylvania.  Also, each summer, the PI's department runs an NSF-funded REU
%  (research experience for undergraduates)
% on the ``Science of Software''.
%  At this
% program, places are reserved for students from traditionally
% under-represented areas (e.g. economically challenged regions of the
% state of North Carolina) and/or students from universities
% lacking advanced research  facilities.
% Some of the simpler data mining concepts for this proposal would be suitable for lectures and REU projects.



%\input{f/under}
\subsection{Dissemination of Knowledge}
As mentioned above, much of the work involved with this work centers around community outreach and encourages a large number of other teams to use {\IT}.

Apart from that, the PI and co-Pi have  an extensive track record of publishing at senior SE venues and so, it should be expected that the results of this work will be widely visible.

Also, as mentioned above, the PI has a long history of publishing papers along with reproduction packages holding the data and the scripts required to replicated the papers' results. 
All the our methods used here will be based around software tools in widespread   use (Github, Travis CI, etc). 
We will release all our tools via open source licenses so our results can be readily applicable to researcher or developers using Github, etc.
For more on this point, see our Data Management Plan.
\vspace{8pt}