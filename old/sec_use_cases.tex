\note{Dave/Ed are moving this content to Section~\ref{sec:research:enabled}}

\paragraph{SLR Authors}
A team of researchers collaborate to plan, conduct, and document a new SLR.  
They coordinate their efforts at each step in the process. 
The members of the research team may be geographically distributed, and work over time.

A team of researchers collaborate to plan, conduct, and document the update of an existing SLR.  
They utilize results of the previous SLR as a starting point for the updated SLR. 
They coordinate their efforts at each step in the process.  
The members of the research team may be geographically distributed.

A researcher wishes to explore the relationships among the entities of a given domain.  
They utilize literature identified in SLRs to construct conceptual models and form hypotheses.  
Next, they utilize data previously extracted from primary studies to test the hypotheses.

\paragraph{PhD Students}
A PhD student reviews existing SLRs to assemble a template which they use to plan, conduct, and document a new SLR.  
The student draws upon previous research to identify sources of literature and possible concepts to explore.  
The student’s supervisor provides feedback as needed at each step of the process.  
The student’s supervisor optionally approves completion of each step of the process.

\paragraph{Tool Authors}
The author of an existing tool interfaces the tool with the repository in order to form a portion of a tool chain for the production of SLRs.  
This requires the infrastructure platform to have plug-in points for process and data interchange. 
The tool author utilizes the API of the repository to create, read, and update information in the repository regarding the planning, execution, and documenting of a SLR.

A tool author desires to implement a specific workflow for a step in the SLR process.  
The tool author utilizes the API of the repository to enable access to SLR planning, execution, and documentation information employed in the process being implemented.  
The tool author utilizes the API of the repository to provide input and updates to the information regarding the planning, execution, and documentation of a SLR in which the process being implemented is employed.

\paragraph{Practitioners/Users}
A SLR user searches for information regarding best practices regarding a domain of interest.  
Having operational knowledge of the domain of interest, the user provides feedback (i.e. newly developing trends, possible stratifications in the models, etc.) that assists in guiding future updates of the SLR and new areas of research. 

An industry professional serves as a subject matter expert in an area of study.  
The expert provides guidance and suggestions to research teams preparing SLRs.  
The expert serves as a reviewer of completed SLRs.
